In this section we describe the unification algorithm that we use to solve sets of equations between expressions.
In general, there are three possible outcomes of a unification problem.
The unifier can solve the unification problem and yield a unifier as a definite proof that the two terms can be unified.
The unifier can also find definite evidence that the two expressions cannot be unified.
A third possibility is that the unification algorithm gives up, which is always a possibility because the underlying problem of higher-order unification is not decidable in the general case.

Unification is stateful algorithm, and uses a state $\mathcal{U}$ to keep track of various metadata.

\[
\begin{array}{lclr}
    \mathcal{C} & \Coloneqq & e = e & \emph{Constraint} \\
    \mathcal{\theta} & \Coloneqq & \overline{x \mapsto e} & \emph{Unifier} \\
    \mathcal{U} & \Coloneqq & \{ \text{eqns} : \overline{\mathcal{C}}, \text{cache} :  \overline{\mathcal{C}}, \text{result} : \theta \} & \emph{Unifier State}
\end{array}
\]

%%
%% Subsec: Helper Functions
%%
\subsection{Helper Functions}
\label{subsec:unification:helper}

Todo: Define adding new constraints and adding bindings to the unifier.

%%
%% Subsec: Solving a Single Constraint
%%
\subsection{Solving a Single Constraint}
\label{subsec:unification:solve}

We write $(\mathcal{U}, \mathcal{C}) \mapsto \mathcal{U}$ for a single step of the constraint solver which solves the constraint C in state U and yields the new state U.

\begin{align*}
    \solveconstraint{e = e'} & \coloneq \mathcal{U} \quad \text{(if $e \equiv_\alpha e'$)} \\
    \solveconstraint{x = e} & \coloneq \addbinding{x}{e} \\
    \solveconstraint{e = x} & \coloneq \addbinding{x}{e} \\
    \solveconstraint{\Type = \Type} & \coloneq \mathcal{U} \\
    \solveconstraint{\mathcal{T}_1\sigma_1 = \mathcal{T}_2 \sigma_2} & \coloneq \ldots \\
    \solveconstraint{K_1\sigma_1 = K_2 \sigma_2} & \coloneq \ldots \\
    \solveconstraint{e.d\sigma = e.d\sigma} & \coloneqq \ldots \\
\end{align*}