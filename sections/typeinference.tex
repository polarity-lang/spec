In this section we show how to typecheck constructors applied to arguments.
We first specify the rules for the system without implicit arguments, and then for the system with implicit arguments.

\subsection{Type Inference Without Implicit Arguments}

Let us first recapitulate how the rules for constructors without implicit arguments work.

\subsubsection*{Declarative Rules}

We start with the rule \textsc{Data} which checks whether a given data type declaration is wellformed:
\begin{prooftree}
    \AxiomC{$\vdash \telescope{\Psi}$}
    \AxiomC{$\forall i : \quad \vdash \telescope{\Xi_i} \quad \Xi_i \vdash \rho_i : \Psi$}
    \RightLabel{\textsc{Data}}
    \BinaryInfC{$\mathbf{data}\ \mathcal{T}\Psi\ \{\ldots, \mathcal{K}_i \Xi_i : \mathcal{T}\rho_i, \ldots \}$}
\end{prooftree}
In this rule we introduce a new data type $\mathcal{T}$ with parameters/indices $\Psi$.
Every constructor $\mathcal{K}_i$ of that data type expects a list of arguments which are specified by the telescope $\Xi$ and returns a specific instance $\mathcal{T}\rho_i$ of the data type that is specified by the substitution $\rho_i$ which must typecheck in the context $\Xi_i$.
In the following rules we always assume that this data definition is implicitly in scope when we specify the rules for constructors.
This convention allows us to write one premiss less in each of the rules; in a real implementation we have to look up in the program if a data type declaration for the constructor is in scope.


The rule for typing a constructor $\mathcal{K}_i$ then looks as follows.

\begin{prooftree}
    \AxiomC{$\Gamma \vdash \sigma : \Xi_i$}
    \RightLabel{\textsc{Ctor}}
    \UnaryInfC{$\Gamma \vdash \mathcal{K}_i\sigma : \mathcal{T}\rho_i[\sigma / \Xi_i]$}
\end{prooftree}

In this rule we have to guarantee that the concrete arguments $\sigma$ which are passed to the constructor $\mathcal{K}_i$ correspond to the types of the arguments $\Xi_i$ that were specified in the data type declaration.
The resulting return type is then a specialization of the type $\mathcal{T}\rho_i$ of the data type declaration which we obtain by substituting the concrete arguments for the variables in $\Xi_i$.

\subsubsection*{Bidirectional Rules}
We now introduce the bidirectional rules for this system.
For notation, we use $\ldots \Rightarrow \ldots$ for inference and $\ldots \Leftarrow \ldots$ for checking.
The rule for data declarations is mostly unchanged, except that we now explicitly check the $\rho_i$:

\begin{prooftree}
    \AxiomC{$\vdash \telescope{\Psi}$}
    \AxiomC{$\forall i : \quad \vdash \telescope{\Xi_i} \quad \Xi_i \vdash \rho_i \Leftarrow \Psi$}
    \RightLabel{\textsc{Data}}
    \BinaryInfC{$\mathbf{data}\ \mathcal{T}\Psi\ \{\ldots, \mathcal{K}_i \Xi_i : \mathcal{T}\rho_i, \ldots \}$}
\end{prooftree}

The rules for constructors are more interesting, since we now have two rules:

\begin{prooftree}
    \AxiomC{$\Gamma \vdash \sigma \Leftarrow \Xi_i$}
    \RightLabel{\textsc{Ctor-Infer}}
    \UnaryInfC{$\Gamma \vdash \mathcal{K}_i\sigma \Rightarrow \mathcal{T}\rho_i[\sigma / \Xi_i]$}
\end{prooftree}

\begin{prooftree}
    \AxiomC{$\Gamma \vdash \sigma \Leftarrow \Xi_i$}
    \AxiomC{$\unify(\tau, \mathcal{T}\rho_i[\sigma / \Xi_i])$}
    \RightLabel{\textsc{Ctor-Check}}
    \BinaryInfC{$\Gamma \vdash \mathcal{K}_i\sigma \Leftarrow \tau$}
\end{prooftree}

In the rule \textsc{Ctor-Infer} we can \emph{check} the arguments of the constructor since we know from the data definition what its types must be.
This is also the case for \textsc{Ctor-Check}, but we now have an additional step where we use unification.

\subsection{Type Inference With Implicit Arguments}

We first have to modify the rule for data types in the following way.

\begin{prooftree}
    \AxiomC{$\vdash \telescope{\Psi}$}
    \AxiomC{$\forall i : \quad \vdash \telescope{\Xi_i,\Xi'_i,} \quad \Xi_i,\Xi'_i \vdash \rho_i \Leftarrow \Psi$}
    \RightLabel{\textsc{Data}}
    \BinaryInfC{$\mathbf{data}\ \mathcal{T}\Psi\ \{\ldots, \mathcal{K}_i \{\Xi_i\}(\Xi'_i) : \mathcal{T}\rho_i, \ldots \}$}
\end{prooftree}

For the moment we don't allow implicit arguments for type constructors.
But every constructor comes with a list of implicit arguments and a list of explicit arguments.

\begin{prooftree}
    \AxiomC{$\sigma\ \mathrm{fresh}$}
    \AxiomC{$\Gamma \vdash (\sigma,\sigma') \Leftarrow \Xi_i, \Xi'_i$}
    \RightLabel{\textsc{Ctor-Infer}}
    \BinaryInfC{$\Gamma \vdash \mathcal{K}_i\{\_\}(\sigma') \Rightarrow \mathcal{T}\rho_i[\sigma,\sigma' / \Xi_i, \Xi'_i]$}
\end{prooftree}

\begin{prooftree}
    \AxiomC{$\sigma\ \mathrm{fresh}$}
    \AxiomC{$\Gamma \vdash (\sigma,\sigma') \Leftarrow \Xi_i, \Xi'_i$}
    \AxiomC{$\unify(\tau, \mathcal{T}\rho_i[\sigma,\sigma' / \Xi_i, \Xi'_i])$}
    \RightLabel{\textsc{Ctor-Check}}
    \TrinaryInfC{$\Gamma \vdash \mathcal{K}_i\{\_\}(\sigma') \Leftarrow \tau$}
\end{prooftree}

% \subsection{Checking rules}
% \label{subsec:typeinference:check}

% \begin{prooftree}
%     \AxiomC{$\normalize{t}{T\rho}$}
%     \AxiomC{$\mathbf{codata}\ T\Xi \{\} \in \Theta$}
%     \AxiomC{$\Gamma \vdash \rho \Leftarrow \Xi$}
%     \AxiomC{$\Gamma; T\rho \vdash o$}
%     \RightLabel{\textsc{Cocase}}
%     \QuaternaryInfC{$\Gamma \vdash \mathbf{cocase}\ \{ \overline{o} \} \Leftarrow t$}
% \end{prooftree}

% \begin{prooftree}
%     \AxiomC{$\mathbf{codata}\ T\Xi \{ (z: T\sigma).D\Xi : t \} \in \Theta$}
%     \AxiomC{$\unify(\rho,\sigma) = \psi$}
%     \AxiomC{$\Gamma\psi \vdash \ldots$}
%     \RightLabel{\textsc{Cocase-Ok}}
%     \TrinaryInfC{$\Gamma; T\rho \vdash D\Delta \mapsto e$}
% \end{prooftree}

% \begin{prooftree}
%     \AxiomC{$\mathbf{codata}\ T\Xi \{ (z: T\sigma).D\Xi : t \} \in \Theta$}
%     \AxiomC{$\unify(\rho,\sigma) = \text{fail}$}
%     \RightLabel{\textsc{Cocase-Absurd}}
%     \BinaryInfC{$\Gamma; T\rho \vdash D\Delta\ \mathbf{absurd}$}
% \end{prooftree}

% \subsection{Inference rules}
% \label{subsec:typeinference:infer}

% \begin{prooftree}
%     \AxiomC{$(x : t) \in \Gamma$}
%     \RightLabel{\textsc{Var}}
%     \UnaryInfC{$\Gamma \vdash x : t$}
% \end{prooftree}

% \begin{prooftree}
%     \AxiomC{}
%     \RightLabel{\textsc{Type}}
%     \UnaryInfC{$\Gamma \vdash \Type \Rightarrow \Type$}
% \end{prooftree}

% \begin{prooftree}
%     \AxiomC{$\mathbf{data}\ T\Delta\ \{\} \in \Theta$}
%     \AxiomC{$\Gamma \vdash \sigma \Leftarrow \Delta$}
%     \RightLabel{\textsc{TyCtor}$_1$}
%     \BinaryInfC{$\Gamma \vdash T\sigma \Rightarrow \Type$}
% \end{prooftree}

% \begin{prooftree}
%     \AxiomC{$\mathbf{codata}\ T\Delta\ \{\} \in \Theta$}
%     \AxiomC{$\Gamma \vdash \sigma \Leftarrow \Delta$}
%     \RightLabel{\textsc{TyCtor}$_2$}
%     \BinaryInfC{$\Gamma \vdash T\sigma \Rightarrow \Type$}
% \end{prooftree}

% \begin{prooftree}
%     \AxiomC{$\Gamma \vdash e \Leftarrow t$}
%     \RightLabel{\textsc{Annotation}}
%     \UnaryInfC{$\Gamma \vdash (e : t) \Rightarrow t$}
% \end{prooftree}

% \begin{prooftree}
%     \AxiomC{$\mathbf{data}\ T\Delta\ \{ K\Xi : T\rho \} \in \Theta$}
%     \AxiomC{$\Gamma \vdash \sigma \Leftarrow \Xi$}
%     \RightLabel{\textsc{Ctor}}
%     \BinaryInfC{$\Gamma \vdash K\sigma \Rightarrow T\rho[\sigma/\Xi]$}
% \end{prooftree}

% \begin{prooftree}
%     \AxiomC{$\mathbf{codata}\ T\Delta\ \{ (z : T\rho).D\Xi : t\} \in \Theta$}
%     \AxiomC{$\Gamma \vdash \sigma \Leftarrow \Xi$}
%     \AxiomC{$\Gamma \vdash e \Leftarrow T\rho[\sigma / \Xi]$}
%     \RightLabel{\textsc{Dtor}}
%     \TrinaryInfC{$\Gamma \vdash e.D\sigma \Rightarrow t[\sigma/\Xi][e/z]$}
% \end{prooftree}