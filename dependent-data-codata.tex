\documentclass[acmsmall,review,anonymous,nonacm]{acmart}
\settopmatter{printfolios=true,printccs=false,printacmref=true}

%%
%% Journal information
%%
\acmJournal{PACMPL}
\acmVolume{1}
\acmNumber{OOPSLA}
\acmArticle{1}
\acmYear{2020}
\acmMonth{1}
\acmDOI{}
\startPage{1}

%%
%% Copyright information
%%
\setcopyright{none}

%%
%% Bibliography and Citation Style
%%
\bibliographystyle{ACM-Reference-Format}
\citestyle{acmauthoryear}

%%
%% Imported Packages
%%
\usepackage{booktabs}
\usepackage{subcaption}
\usepackage{amsmath}
\usepackage[all,cmtip]{xy}
\usepackage[nameinlink,noabbrev,capitalise]{cleveref}
\usepackage{csquotes}
\usepackage{booktabs}
\usepackage{microtype}
\usepackage{stmaryrd}
\usepackage{graphicx}
\usepackage{enumitem}
\usepackage{mathtools}
\usepackage{courier}
\usepackage{bussproofs}
\usepackage{tikz-cd}
\usepackage{diagbox}
\usepackage{afterpage}
\usepackage[normalem]{ulem}
\usepackage{dashbox}
\usepackage{mdframed}
\usepackage[mathscr]{euscript}
\DeclareMathAlphabet\mathbfscr{U}{eus}{b}{n}
\usepackage{colortbl}
\usepackage{anyfontsize}
\usepackage{alltt} % verbatim with colors
\usepackage{wrapfig}
\usepackage{thm-restate} % restatable environment
%%
%% Todo notes
%%
\setlength{\marginparwidth}{2cm} % This is needed, otherwise the package "todonotes" will generate a warning.
\usepackage[]{todonotes}
\newcommand{\klaus}[2][]{\todo[inline,color=orange,author=Klaus,#1]{#2}}
\newcommand{\david}[2][]{\todo[inline,color=red!40,author=David,#1]{#2}}
\newcommand{\tim}[2][]{\todo[inline,color=yellow!40,author=Tim,#1]{#2}}

\usepackage{syntax}

%%
%% Hyphenations and Macros
%%
\input{macros/hyphenations.tex}
%%%%%%%%%%%%%%%%%%%%%%%%%%%%%%%%%%%%%%%%%%%%%%%%%%%%%%%%%%%%%%%%%%%%%%%%%%%%%%%
%% Syntax
%%%%%%%%%%%%%%%%%%%%%%%%%%%%%%%%%%%%%%%%%%%%%%%%%%%%%%%%%%%%%%%%%%%%%%%%%%%%%%%
\newcommand{\Type}{\text{Type}}


%%%%%%%%%%%%%%%%%%%%%%%%%%%%%%%%%%%%%%%%%%%%%%%%%%%%%%%%%%%%%%%%%%%%%%%%%%%%%%%
%% Normalization
%%%%%%%%%%%%%%%%%%%%%%%%%%%%%%%%%%%%%%%%%%%%%%%%%%%%%%%%%%%%%%%%%%%%%%%%%%%%%%%
\newcommand{\normalize}[2]{#1 \leadsto #2}

%%%%%%%%%%%%%%%%%%%%%%%%%%%%%%%%%%%%%%%%%%%%%%%%%%%%%%%%%%%%%%%%%%%%%%%%%%%%%%%
%% Unification
%%%%%%%%%%%%%%%%%%%%%%%%%%%%%%%%%%%%%%%%%%%%%%%%%%%%%%%%%%%%%%%%%%%%%%%%%%%%%%%
\newcommand{\unify}{\text{unify}}

% Color definitions

\usepackage{xcolor}
% Color definitions for XFN
\definecolor{xfnBlack}{rgb}{0,0,0}
\definecolor{xfnBlue}{rgb}{0.06, 0.2, 0.65}
\definecolor{xfnGreen}{RGB}{0,155,85}
\definecolor{xfnRed}{rgb}{0.8,0.4,0.3}
\definecolor{xfnCyan}{rgb}{0.0, 0.5, 1.0}
\definecolor{xfnMagenta}{rgb}{0.8, 0.13, 0.13}
\definecolor{xfnYellow}{rgb}{0.91, 0.84, 0.42}
\definecolor{xfnWhite}{rgb}{1,1,1}

% The \xfnBackslash command needs to ignore its argument and produce a backslash
\newcommand{\xfnBackslash}[1]{\textbackslash{}}


\begin{document}

%%
%% Title information
%%
\title{Type Inference for Dependent Data and Codata Types}

%%
%% Keywords
%%
\keywords{Dependent Types, Type Inference, Implicit Arguments}

%%
%% CCS Classification
%%

\begin{CCSXML}
  <ccs2012>
  <concept>
  <concept_id>10003752.10003753.10003754.10003733</concept_id>
  <concept_desc>Theory of computation~Lambda calculus</concept_desc>
  <concept_significance>300</concept_significance>
  </concept>
  <concept>
  <concept_id>10003752.10003790.10011740</concept_id>
  <concept_desc>Theory of computation~Type theory</concept_desc>
  <concept_significance>300</concept_significance>
  </concept>
  </ccs2012>
\end{CCSXML}

\ccsdesc[300]{Theory of computation~Lambda calculus}
\ccsdesc[300]{Theory of computation~Type theory}

%%
%% Author: David Binder
%%
\author{David Binder}
\orcid{0000-0003-1272-0972}
\affiliation{
  \department{Department of Computer Science}
  \institution{University of Tübingen}
  \streetaddress{Sand 14}
  \city{Tübingen}
  \postcode{72076}
  \country{Germany}
}
\email{david.binder@uni-tuebingen.de}


%%
%% Author: Tim Süberkrüb
%%
\author{Tim Süberkrüb}
\orcid{0000-0001-8709-6321}
\affiliation{
  \department{Department of Computer Science}
  \institution{University of Tübingen}
  \streetaddress{Sand 14}
  \city{Tübingen}
  \postcode{72076}
  \country{Germany}
}
\email{tim.sueberkrueb@aleph-alpha.de}


%%
%% Abstract
%%
\begin{abstract}
  Most presentations of dependent type theories are based on a builtin set of types, including the function type.
  We present a type theory which is based on dependent data and codata types and show how to infer types for this system.
\end{abstract}

\maketitle

%%
%% Introduction
%%
\section{Introduction}
\label{sec:intro}
In this short development we describe how to infer the implicit arguments to constructors.
% \subsection{Overview}

% \begin{itemize}
%     \item In \cref{sec:syntax} we present the syntax of expressions.
%     \item In \cref{sec:normalization} we present the untyped normalization-by-evaluation algorithm used to normalize terms.
%     \item In \cref{sec:unification} we present the unification algorithm which computes unifications.
%     \item In \cref{sec:typeinference} we present the bidirectional type inference algorithm used to check and infer types.
%     \item In \cref{sec:relatedwork} we discuss related work.
% \end{itemize}

\section{Syntax}
\label{sec:syntax}
\[
\begin{array}{lclr}
    \mathcal{T} &\in &\textsc{TypeNames} & \\
    \mathcal{K} &\in &\textsc{ConstructorNames} & \\
    \mathcal{D} &\in &\textsc{DestructorNames} & \\
    x,y,z &\in &\textsc{Variables} & \\[0.15cm]
    % Expressions
    \rho, \sigma &\Coloneqq & () \mid (e,\sigma) & \emph{Substitution} \\
    e,s,t &\Coloneqq & x & \emph{Variable} \\
    & \mid & \Type & \emph{Universe} \\
    & \mid & \mathcal{T}\sigma & \emph{Type constructor} \\
    & \mid & \mathcal{K}\sigma & \emph{Producer} \\
    & \mid & e.\mathcal{D}\sigma & \emph{Consumer} \\
    & \mid & \mathbf{case}\ e\ \mathbf{as}\ x\ \mathbf{return}\ e\  \mathbf{of}\ \{ \overline{a} \} & \emph{Case} \\
    & \mid & \mathbf{cocase}\ \mathbf{rec}\ x\ \{ \overline{o} \} & \emph{Cocase} \\
    & \mid & e : e & \emph{Annotation} \\
    & \mid & ? & \emph{Hole} \\
    a & \Coloneqq & K\Delta \mapsto e &\emph{Case} \\
    & \mid & K\Delta\ \mathbf{absurd} &\emph{Absurd case} \\
    o & \Coloneqq & D\Delta \mapsto e &\emph{Cocase} \\
    & \mid & D\Delta\ \mathbf{absurd} & \emph{Absurd cocase} \\
\end{array}
\]

\section{Normalization}
\label{sec:normalization}
We describe the implementation of the function $\normalize{e}{e'}$, which says that the term $e$ normalizes to the term $e'$.

\section{Unification}
\label{sec:unification}
We describe the function $\unify$.
There are three possible outcomes for unification:
\begin{enumerate}
    \item The set of equations can be unified, yielding a mgu: 
    \item The set of equations can definitely not be unified, yielding a fail.
    \item The unifier gives up.
\end{enumerate}

\section{Type Inference}
\label{sec:typeinference}
In this section we show how to typecheck constructors applied to arguments.
We first specify the rules for the system without implicit arguments, and then for the system with implicit arguments.

\subsection{Type Inference Without Implicit Arguments}

Let us first recapitulate how the rules for constructors without implicit arguments work.

\subsubsection*{Declarative Rules}

We start with the rule \textsc{Data} which checks whether a given data type declaration is wellformed:
\begin{prooftree}
    \AxiomC{$\vdash \telescope{\Psi}$}
    \AxiomC{$\forall i : \quad \vdash \telescope{\Xi_i} \quad \Xi_i \vdash \rho_i : \Psi$}
    \RightLabel{\textsc{Data}}
    \BinaryInfC{$\mathbf{data}\ \mathcal{T}\Psi\ \{\ldots, \mathcal{K}_i \Xi_i : \mathcal{T}\rho_i, \ldots \}$}
\end{prooftree}
In this rule we introduce a new data type $\mathcal{T}$ with parameters/indices $\Psi$.
Every constructor $\mathcal{K}_i$ of that data type expects a list of arguments which are specified by the telescope $\Xi$ and returns a specific instance $\mathcal{T}\rho_i$ of the data type that is specified by the substitution $\rho_i$ which must typecheck in the context $\Xi_i$.
In the following rules we always assume that this data definition is implicitly in scope when we specify the rules for constructors.
This convention allows us to write one premiss less in each of the rules; in a real implementation we have to look up in the program if a data type declaration for the constructor is in scope.


The rule for typing a constructor $\mathcal{K}_i$ then looks as follows.

\begin{prooftree}
    \AxiomC{$\Gamma \vdash \sigma : \Xi_i$}
    \RightLabel{\textsc{Ctor}}
    \UnaryInfC{$\Gamma \vdash \mathcal{K}_i\sigma : \mathcal{T}\rho_i[\sigma / \Xi_i]$}
\end{prooftree}

In this rule we have to guarantee that the concrete arguments $\sigma$ which are passed to the constructor $\mathcal{K}_i$ correspond to the types of the arguments $\Xi_i$ that were specified in the data type declaration.
The resulting return type is then a specialization of the type $\mathcal{T}\rho_i$ of the data type declaration which we obtain by substituting the concrete arguments for the variables in $\Xi_i$.

\subsubsection*{Bidirectional Rules}
We now introduce the bidirectional rules for this system.
For notation, we use $\ldots \Rightarrow \ldots$ for inference and $\ldots \Leftarrow \ldots$ for checking.
The rule for data declarations is mostly unchanged, except that we now explicitly check the $\rho_i$:

\begin{prooftree}
    \AxiomC{$\vdash \telescope{\Psi}$}
    \AxiomC{$\forall i : \quad \vdash \telescope{\Xi_i} \quad \Xi_i \vdash \rho_i \Leftarrow \Psi$}
    \RightLabel{\textsc{Data}}
    \BinaryInfC{$\mathbf{data}\ \mathcal{T}\Psi\ \{\ldots, \mathcal{K}_i \Xi_i : \mathcal{T}\rho_i, \ldots \}$}
\end{prooftree}

The rules for constructors are more interesting, since we now have two rules:

\begin{prooftree}
    \AxiomC{$\Gamma \vdash \sigma \Leftarrow \Xi_i$}
    \RightLabel{\textsc{Ctor-Infer}}
    \UnaryInfC{$\Gamma \vdash \mathcal{K}_i\sigma \Rightarrow \mathcal{T}\rho_i[\sigma / \Xi_i]$}
\end{prooftree}

\begin{prooftree}
    \AxiomC{$\Gamma \vdash \sigma \Leftarrow \Xi_i$}
    \AxiomC{$\unify(\tau, \mathcal{T}\rho_i[\sigma / \Xi_i])$}
    \RightLabel{\textsc{Ctor-Check}}
    \BinaryInfC{$\Gamma \vdash \mathcal{K}_i\sigma \Leftarrow \tau$}
\end{prooftree}

In the rule \textsc{Ctor-Infer} we can \emph{check} the arguments of the constructor since we know from the data definition what its types must be.
This is also the case for \textsc{Ctor-Check}, but we now have an additional step where we use unification.

\subsection{Type Inference With Implicit Arguments}

We first have to modify the rule for data types in the following way.

\begin{prooftree}
    \AxiomC{$\vdash \telescope{\Psi}$}
    \AxiomC{$\forall i : \quad \vdash \telescope{\Xi_i,\Xi'_i,} \quad \Xi_i,\Xi'_i \vdash \rho_i : \Psi$}
    \RightLabel{\textsc{Data}}
    \BinaryInfC{$\mathbf{data}\ \mathcal{T}\Psi\ \{\ldots, \mathcal{K}_i \{\Xi_i\}(\Xi'_i) : \mathcal{T}\rho_i, \ldots \}$}
\end{prooftree}

For the moment we don't allow implicit arguments for type constructors.
But every constructor comes with a list of implicit arguments and a list of explicit arguments.

\begin{prooftree}
    \AxiomC{$\sigma\ \mathrm{fresh}$}
    \AxiomC{$\Gamma \vdash (\sigma,\sigma') \Leftarrow \Xi_i, \Xi'_i$}
    \RightLabel{\textsc{Ctor-Infer}}
    \BinaryInfC{$\Gamma \vdash \mathcal{K}_i\{\_\}(\sigma') \Rightarrow \mathcal{T}\rho_i[\sigma,\sigma' / \Xi_i, \Xi'_i]$}
\end{prooftree}

\begin{prooftree}
    \AxiomC{$\sigma\ \mathrm{fresh}$}
    \AxiomC{$\Gamma \vdash (\sigma,\sigma') \Leftarrow \Xi_i, \Xi'_i$}
    \AxiomC{$\unify(\tau, \mathcal{T}\rho_i[\sigma,\sigma' / \Xi_i, \Xi'_i])$}
    \RightLabel{\textsc{Ctor-Check}}
    \TrinaryInfC{$\Gamma \vdash \mathcal{K}_i\{\_\}(\sigma') \Leftarrow \tau$}
\end{prooftree}

% \subsection{Checking rules}
% \label{subsec:typeinference:check}

% \begin{prooftree}
%     \AxiomC{$\normalize{t}{T\rho}$}
%     \AxiomC{$\mathbf{codata}\ T\Xi \{\} \in \Theta$}
%     \AxiomC{$\Gamma \vdash \rho \Leftarrow \Xi$}
%     \AxiomC{$\Gamma; T\rho \vdash o$}
%     \RightLabel{\textsc{Cocase}}
%     \QuaternaryInfC{$\Gamma \vdash \mathbf{cocase}\ \{ \overline{o} \} \Leftarrow t$}
% \end{prooftree}

% \begin{prooftree}
%     \AxiomC{$\mathbf{codata}\ T\Xi \{ (z: T\sigma).D\Xi : t \} \in \Theta$}
%     \AxiomC{$\unify(\rho,\sigma) = \psi$}
%     \AxiomC{$\Gamma\psi \vdash \ldots$}
%     \RightLabel{\textsc{Cocase-Ok}}
%     \TrinaryInfC{$\Gamma; T\rho \vdash D\Delta \mapsto e$}
% \end{prooftree}

% \begin{prooftree}
%     \AxiomC{$\mathbf{codata}\ T\Xi \{ (z: T\sigma).D\Xi : t \} \in \Theta$}
%     \AxiomC{$\unify(\rho,\sigma) = \text{fail}$}
%     \RightLabel{\textsc{Cocase-Absurd}}
%     \BinaryInfC{$\Gamma; T\rho \vdash D\Delta\ \mathbf{absurd}$}
% \end{prooftree}

% \subsection{Inference rules}
% \label{subsec:typeinference:infer}

% \begin{prooftree}
%     \AxiomC{$(x : t) \in \Gamma$}
%     \RightLabel{\textsc{Var}}
%     \UnaryInfC{$\Gamma \vdash x : t$}
% \end{prooftree}

% \begin{prooftree}
%     \AxiomC{}
%     \RightLabel{\textsc{Type}}
%     \UnaryInfC{$\Gamma \vdash \Type \Rightarrow \Type$}
% \end{prooftree}

% \begin{prooftree}
%     \AxiomC{$\mathbf{data}\ T\Delta\ \{\} \in \Theta$}
%     \AxiomC{$\Gamma \vdash \sigma \Leftarrow \Delta$}
%     \RightLabel{\textsc{TyCtor}$_1$}
%     \BinaryInfC{$\Gamma \vdash T\sigma \Rightarrow \Type$}
% \end{prooftree}

% \begin{prooftree}
%     \AxiomC{$\mathbf{codata}\ T\Delta\ \{\} \in \Theta$}
%     \AxiomC{$\Gamma \vdash \sigma \Leftarrow \Delta$}
%     \RightLabel{\textsc{TyCtor}$_2$}
%     \BinaryInfC{$\Gamma \vdash T\sigma \Rightarrow \Type$}
% \end{prooftree}

% \begin{prooftree}
%     \AxiomC{$\Gamma \vdash e \Leftarrow t$}
%     \RightLabel{\textsc{Annotation}}
%     \UnaryInfC{$\Gamma \vdash (e : t) \Rightarrow t$}
% \end{prooftree}

% \begin{prooftree}
%     \AxiomC{$\mathbf{data}\ T\Delta\ \{ K\Xi : T\rho \} \in \Theta$}
%     \AxiomC{$\Gamma \vdash \sigma \Leftarrow \Xi$}
%     \RightLabel{\textsc{Ctor}}
%     \BinaryInfC{$\Gamma \vdash K\sigma \Rightarrow T\rho[\sigma/\Xi]$}
% \end{prooftree}

% \begin{prooftree}
%     \AxiomC{$\mathbf{codata}\ T\Delta\ \{ (z : T\rho).D\Xi : t\} \in \Theta$}
%     \AxiomC{$\Gamma \vdash \sigma \Leftarrow \Xi$}
%     \AxiomC{$\Gamma \vdash e \Leftarrow T\rho[\sigma / \Xi]$}
%     \RightLabel{\textsc{Dtor}}
%     \TrinaryInfC{$\Gamma \vdash e.D\sigma \Rightarrow t[\sigma/\Xi][e/z]$}
% \end{prooftree}

\section{Related Work}
\label{sec:relatedwork}
\cite{Cockx2017dependent}

\section{Conclusion}
\label{sec:conclusion}
\input{sections/conclusion.tex}


%%
%% Bibliography
%%
\bibliography{bibliography/bibliography.bib,bibliography/ownpublications.bib}

\appendix

\end{document}
